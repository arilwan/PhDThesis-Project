\chapter*{Resumo}
\addcontentsline{toc}{chapter}{Resumo}

O rápido ritmo de urbanização e o contínuo crescimento da população global estão a exercer uma pressão significativa sobre os sistemas de transporte, exigindo o desenvolvimento de soluções inteligentes e rentáveis para enfrentar os desafios emergentes. Uma das vias promissoras para melhorar a eficiência do transporte reside na utilização eficaz das vastas quantidades de dados espaço-temporais gerados de forma colaborativa por dispositivos móveis, como smartphones, transportados pelos indivíduos. Além disso, os avanços nas ferramentas de mineração de dados e nas técnicas de aprendizagem automática têm-se revelado fundamentais para analisar esses dados, permitindo a extração de informações valiosas sobre os padrões de mobilidade.

Os dados de trajetórias GPS constituem um recurso crucial para investigar os padrões de mobilidade humana e avaliar a sua influência nos sistemas de transporte. No entanto, um desafio significativo reside na determinação precisa do modo de transporte para cada trajeto, uma vez que as trajetórias GPS não contêm informações explícitas sobre o modo de viagem. Assim, esta tese tem como objetivo abordar esta limitação através de uma abordagem de aprendizagem supervisionada, englobando a análise de características e técnicas-chave, e propondo, em última instância, abordagens inovadoras para identificar modos de transporte a partir de conjuntos de dados de trajetórias GPS. A realização bem-sucedida deste objetivo exige uma compreensão aprofundada e um pré-processamento minucioso dos dados brutos de trajetórias.

Para este fim, esta tese investiga a eficácia de várias técnicas de engenharia de características para extrair informações relevantes de trajetórias GPS. Além disso, propõe um novo método de seleção de características de subespaço de classe que identifica as características mais discriminativas para cada modo de transporte, aumentando assim a precisão e robustez da detecção de modo. Além disso, é proposto um framework de classificação hierárquica, aproveitando os erros de classificação de um classificador plano padrão para melhorar o desempenho geral da classificação. A tese também analisa o impacto do desequilíbrio de classes no desempenho dos classificadores de detecção de modo de transporte (TMD) e explora estratégias para mitigar seus efeitos.

Finalmente, o desempenho dos métodos e modelos propostos é avaliado em conjuntos de dados de trajetória GPS do mundo real. Os resultados experimentais demonstram a superioridade do framework proposto para a previsão correta do modo de transporte.

\chapter*{Abstract}
\addcontentsline{toc}{chapter}{Abstract}

The rapid pace of urbanisation and the continued growth of the global population are exerting significant pressure on transportation systems, demanding the development of intelligent and cost-effective solutions to address emerging challenges. One promising avenue for enhancing transportation efficiency lies in the effective utilisation of the vast quantities of crowdsourced spatiotemporal data generated by mobile devices, such as smartphones, carried by individuals. Furthermore, advancements in data mining tools and machine learning techniques are proving instrumental in analysing this data, enabling the extraction of valuable insights into mobility patterns.

GPS trajectory data constitutes a crucial resource for investigating human mobility patterns and evaluating their influence on transportation systems. A significant challenge, however, lies in accurately determining the mode of transport for each journey, as GPS trajectories lack explicit travel mode information. Accordingly, this thesis aims to address this limitation through a supervised learning approach, encompassing the analysis of key features and techniques, and ultimately proposing novel approaches for identifying transportation modes from GPS trajectory datasets. The successful realisation of this objective necessitates a comprehensive understanding and meticulous pre-processing of the raw trajectory data.


To this end, this thesis investigates the effectiveness of various feature engineering techniques for extracting relevant information from GPS trajectories. Furthermore, it proposes a novel class-subspace feature selection method that identifies the most discriminative features for each transportation mode, thereby enhancing the accuracy and robustness of mode detection. Additionally, a hierarchical classification framework is proposed, leveraging the misclassification errors of a standard flat classifier to improve overall classification performance. The thesis also analyses the impact of class imbalance on the performance of transportation mode detection (TMD) classifiers and explores strategies to mitigate its effects. 

Finally, the performance of the proposed methods and models are evaluated on real-world GPS trajectory datasets. The experimental results demonstrate the superiority of the proposed framework towards correct prediction of transportation mode.

