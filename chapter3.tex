\chapter{Datasets and Preprocessing Tasks} \label{chap3}

The previous chapter reviewed the state-of-the-art in transportation mode detection, including discussions of various data types, approaches, and algorithms employed in the literature. This chapter delves into the datasets utilised in this thesis research, providing detailed characterisations of each.

\section{Datasets Description}
Two real-world GPS trajectory datasets are employed in this thesis, which are detailed.

    \subsection{The SenseMyFEUP Dataset}
    The \emph{SenseMyFEUP (SMF2016)} dataset is a GPS trajectory dataset of 227 participants collected in Porto, Portugal, in April 2016 \cite{rodrigues2017impact}. It was collected using an Android-based mobile application installed on participants' smartphones. The application was designed to automatically record location data whenever user movement was detected and to stop recording when the user ceased moving. SMF2016 is densely represented at approximately one sample per second or every 5 meters. The location data is structured as a tuple containing the following attributes: timestamp, longitude, latitude, altitude, speed, bearing, and GPS accuracy, thus $ L_{i} = <time[t], lon, lat, alt, speed, bearing, \\gps\_acc>$.


    Moreover, the application includes an end-of-trip survey administered after each trip completion, prompting users to indicate their travel mode. (\Cref{fig:survey}). 
    This self-reported data serves as the ground truth for our analysis. In all, \textit{SMF2016} contains travel mode information for five different transportation modes: \textit{bike, bus, car, foot} and \textit{metro}.

    The SMF2016 dataset additionally includes self-reported information on transportation modes. After each trip, the application prompts participants to complete a survey indicating the transportation mode used for their trip (see \Cref{fig:survey}). This self-reported data serves as the ground truth for our research.  The SMF2016 contains transportation mode information for five different modes: bike, bus, car, foot, and metro.
    \begin{figure}[htbp]
        \centering  % width=6cm, height=7cm
        \includegraphics[scale=0.2]{figures/sensemyfeup.jpeg}
        \caption{SMF2016 end of trip survey.}
        \label{fig:survey}
    \end{figure}
    
    \subsection{The Geolife Dataset}
    We also utilise the publicly accessible \textit{Geolife} dataset, developed and published by the Microsoft Research Asia \cite{zheng2008learning, zheng2010understanding, zheng2010geolife}. Collected from 182 participants, primarily in Beijing, China, over a period spanning nearly six years (April 2007–February 2012), the dataset integrates movement information recorded through various GPS loggers and GPS-enabled phones. Data points feature a spatio-temporal granularity of approximately 1–5 seconds or 5–10 meters. Each trajectory consists of a temporally ordered sequence of latitude and longitude coordinates, represented as $<lat, lon, alt, timestamp>$.

    %According to the \textit{Geolife} user guide \cite{zheng2011geolife}, 73 users have labelled their trajectories with travel mode information. However, we found only 69 users with transportation mode information upon examining the raw data. There are about ten different transportation modes \textit{(walk, bike, bus, car, taxi, train, subway, motorcycle, airplane, boat, run)} available in the dataset. 

    According to the Geolife user guide \cite{zheng2011geolife}, 73 users have labelled their trajectories with travel mode information. However, upon examining the raw data, we found that only 69 users have recorded transportation mode information. The dataset encompasses ten different transportation modes (8 land): walk, bike, bus, car, taxi, train, subway, motorcycle, airplane, and boat. The Geolife user guide further recommends combining car and taxi modes into a single category. 

   \Cref{tab:datasets} summarises the key attributes of the two datasets. To further understand user behaviour, \Cref{fig:smf2016-trips-per-user,fig:geolife-trips-per-user} illustrate the distribution of the number of trips per user in both SMF2016 and Geolife, respectively.

%    The SMF2016 dataset reveals a wide range of trip frequencies among users. Approximately 25\% of users have fewer than 10 trips, while 15\% have more than 100 trips. Similarly, the Geolife dataset shows a skewed distribution, with 25\% of users having fewer than 10 trips and 15\% covered more than 150 trips.

    The SMF2016 dataset reveals a wide range of trip frequencies among users. Approximately 25\% of users have fewer than 10 trips, while 15\% have more than 100 trips. Similarly, the Geolife dataset exhibits a skewed distribution, with 25\% of users having fewer than 10 trips and 15\% having more than 150 trips.
    \newcolumntype{Y}{>{\centering\arraybackslash}X}
\begin{table}[htbp]
     \caption{Key Characteristics of the Datasets}
    \label{tab:datasets}
    \centering
    \begin{tabularx}{\textwidth}{@{}lYY@{}}
    \toprule
     & \textbf{SMF2016}  &  \textbf{Geolife}                     \\
    \midrule
    Period of data collection:  & April 2016  & April 2007 - February 2012   \\
    Number of participants:  &   227   & 182 \\
    Travel mode information: & 5 modes & 10 modes \\
    Sampling rate: & 1 sample per second & 1 sample every 3-5 seconds \\
    Number of trajectories:  & 13, 212     & 18,670     \\
    Number of data points:   & 9,657,405   &  24,876,978  \\
    Distance coverage (km):  &  229,564    &  1,292,951   \\
    Total duration (hours):  &  5,082      &  50,176  \\
    Effective days:          &   30        &  11,129    \\
    Accessibility:           &  \textit{private}  & \textit{public}  \\
    \bottomrule     
    \end{tabularx}
\end{table}
%
\begin{figure}[htb]
    \centering
    \begin{minipage}{0.48\textwidth}
        \centering
        \includegraphics[width=\textwidth]{figures/smf2016-trips-per-user.pdf} 
        \caption{SMF2016 - Distribution of trips per user.}
        \label{fig:smf2016-trips-per-user}
    \end{minipage}
   \hspace{1em} %\hfill
    \begin{minipage}{0.48\textwidth}
        \centering
        \includegraphics[width=\textwidth]{figures/geolife-trips-per-user.pdf} %
        \caption{Geolife - Distribution of trips per user.}
        \label{fig:geolife-trips-per-user}
    \end{minipage}
   % \caption{A main caption describing both figures.}
    \label{fig:main-trip-distrib}
\end{figure}

%\section{Exploratory Analysis}


\section{Data Preprocessing}
This Section outlines the process of chaining GPS traces into trips in the datasets.  

\subsection{Trip Preprocessing}
Both GPS trajectory datasets are chronologically ordered. We define a trip as a sequence of user GPS points in session, separated by a stop of at least 30 minutes.  A trip is composed of multiple GPS points, each represented as a tuple containing latitude, longitude, and timestamp information. Our SMF2016 trip chaining algorithm identifies and joins two GPS traces within a session into a single trip based on their timestamps and location, according to the following criteria \cite{rodrigues2017impact}:      %\looseness=-1
\begin{itemize}
    \item The GPS traces are associated with a particular user,
    \item The time interval ($\Delta t$)  between a trace  and its predecessor is no more than 30 minutes,
    \item The distance between a previous trace and the start of a new one is below 200m.
\end{itemize}

\subsection{Trip Filtering}
However, we observed instances where trips exhibited a significant difference in timestamp between successive GPS points beyond the established threshold. Further investigation revealed the large $\Delta t$ between consecutive traces was caused by "unstopped sessions" where data collection continued without recorded stops. We thoroughly searched through all sessions in the dataset and documented the affected sessions in \Cref{tab:gap-sessions}.

We observed instances where trips exhibited a significant difference in timestamp between successive GPS points, exceeding the established threshold. Further investigation revealed that these large time gaps were due to "unstopped sessions," where data collection continued without recorded stops. We conducted a thorough search of all sessions in the dataset and documented the affected sessions in Table \ref{tab:gap-sessions}.
\begin{table}[htbp]
    \centering
    \tabcolsep=25pt
    \caption{Sessions having $\Delta t > 1hr$  between consecutive GPS traces.} 
    \label{tab:gap-sessions}
    \begin{tabular}{@{} l*{1}{c} @{}} %{c|c}
    \toprule
      Number of cases found    &   58 \\
    % \hline
     Number of sessions affected  & 26 \\
     %\hline
      Total number of sessions in the dataset  &  5368 \\
     %\hline
     Number of trips in unstopped sessions  &   25 \\
     %\hline
%     Single mode trips affected       &  15 \\
    \bottomrule
    \end{tabular}
\end{table}

Notably, the trips exhibiting unstopped user sessions originated from trips surpassing the Porto region's boundaries. After excluding these problematic trips and focusing solely on trips within the Porto area, defined by the coordinates  (41.38786, -8.77034) and (41,04928, -8.4253), we still observed some foot trips covering over 10 kilometers (\Cref{fig:trip-filter}a). This deviation from the typical pattern of foot trips in urban areas prompted us to cap foot trips at 8 kilometers or less. This resulted in trip distance distribution shown in  \Cref{fig:trip-filter}b. In \Cref{tab:trips} we present the statistics of filtered trips.% used in this study.
\begin{table}[htbp]
    \centering
    \tabcolsep=12pt
    \caption{Trips filtering statistics.} \label{tab:trips}
    \begin{tabular}{@{} l*{1}{c} @{}} %{c|c}
    \toprule
     Number of trips before filtering &  3730 \\
     Trips with large $\Delta t$ between consecutive traces     &  15 \\
     Number of trips outside the study area  & 764 \\
     Number of trips within Porto area (after filtering)  &   2951 \\
     Total trips (foot mode capped to 8km)   &  2909  \\
    \bottomrule
    \end{tabular}
\end{table}
%
\begin{figure}%[htbp]
    \centering
    \includegraphics[width=1.\textwidth]{figures/trips-subplots.pdf} 
    \caption{Trip distance distribution by travel mode (a) before trip filtering; (b) after filtering to trips covered in Porto region.}
    \label{fig:trip-filter}
\end{figure}

    \subsection{Trip Segmentation}
    To capture the dynamic nature of human movement patterns as closely as possible and enhance the accuracy of the TMD algorithms, we split the trips into alternating sequences of motion and stationary segments \cite{muhammad2021transportation, rodrigues2017impact}. This granular representation allows for a more detailed analysis of movement behaviour and facilitates the identification of transportation mode transitions. However, this segmentation process can introduce a significant number of short segments, potentially degrading the performance of the TMD classifiers  \cite{rodrigues2017impact}.
    We impose a minimum segment length threshold of 50m to mitigate this issue. This ensures only meaningful and informative segments are considered, improving the classifier's ability to distinguish between different transportation modes. The segmentation process begins by identifying stationary segments, representing periods of inactivity. We employ a threshold-based method using \nth{85} percentile of instantaneous GPS speed. Specifically, a stationary segment is established if the user's average speed falls below 0.5m/s for 5 seconds. 

    In the Geolife dataset, user GPS trajectories are partitioned into separate trips based on the time interval between two successive GPS points exceeding the predetermined threshold of 30 minutes \cite{dabiri2018inferring}. Each trip is then further divided into single-mode segments based on transportation mode, with the begin and end timestamps of each segment annotated.

    \subsection{Point-level Attributes}
    To extract meaningful insights, we compute point-level motion attributes for each point in every trajectory, including speed, acceleration, jerk (rate of change of acceleration), and bearing rate. While the SMF2016 dataset provides instantaneous GPS speed for each location point, the Geolife dataset lacks this information. To address this limitation, we estimate speed for Geolife trajectories by calculating the relative geographical distance between consecutive GPS points using the using the Vincenty Formula \cite{vincenty1975direct}  and dividing it by the time elapsed. 
    
    \begin{figure}%[htbp]
        \centering
        \includegraphics[width=0.75\linewidth]{figures/point-bearing.pdf}
        \caption{Visual representation of bearing between successive GPS points in a trajectory.}
        \label{fig:point-bearing}
    \end{figure}
    
    Consider two consecutive GPS points, $P_m, P_n$, as illustrated in \Cref{fig:point-bearing}, we calculate the speed of $P_m$ using \cref{eq_spd}
    \begin{equation}
        S_{P_m}  = \frac{\vn{Vincenty}(P_m, P_n)}{\Delta t} \label{eq_spd}
    \end{equation}

    \Cref{fig:smf2016-speed-per-mode} illustrates the speed distribution for each transportation mode in the SMF2016 dataset. While foot and metro speeds exhibit similar distributions in the lower range (0 - 5 m/s, about 55\% of the trips), a clear divergence is observed at higher speeds. Bus and car modes show similar speed distributions. The distribution of bike speeds is significantly different from that of other modes, particularly at higher speeds.

    Similarly, in the speed distribution of Geolife in \Cref{fig:geolife-speed-per-mode}, bus and car modes exhibit similar speed distributions, particularly in the lower speed range below 10 m/s. The distributions of foot and bike modes also share similarities, particularly in the lower speed range. However, it is evident that there is a noticeable difference in the distribution of metro compared to other transportation modes.

    \begin{figure}[htbp]
    \centering
    \begin{minipage}{0.48\textwidth}
        \centering
        \includegraphics[width=\textwidth]{figures/smf2016-speed.pdf} 
        \caption{SMF2016  speed distribution per transportation mode.}
        \label{fig:smf2016-speed-per-mode}
    \end{minipage}
   \hspace{1em} %\hfill
    \begin{minipage}{0.48\textwidth}
        \centering
        \includegraphics[width=\textwidth]{figures/geolife-speed.pdf} %
        \caption{Geolife speed distribution per transportatio mode.}
        \label{fig:geolife-speed-per-mode}
    \end{minipage}
   % \caption{A main caption describing both figures.}
    \label{fig:main-speed-distrib}
    \end{figure}
    To calculate point-level acceleration ($A_{P_m}$) and jerk ($J_{P_m}$) at point $P_m$, we utilise \cref{eq_acc,eq_jk} respectively as follows:
    
    \begin{equation}
        A_{P_m} = \frac{S_{P_n} - S_{P_m}}{\Delta t} \label{eq_acc}
    \end{equation}
    
    \begin{equation}
        J_{P_m} = \frac{A_{P_n} - A_{P_m}}{\Delta t} \label{eq_jk}
    \end{equation}

    Different transportation modes exhibit varying rates of change in direction. For example, pedestrians and cyclists frequently change their direction, while cars and metro trains tend to follow a more straight (linear) paths. The bearing, which quantifies the angular deviation from the true north \cite{dabiri2018inferring}, of the straight line connecting two consecutive GPS points  (as illustrated in Figure \ref{fig:point-bearing}), provides a measure of the directional change. We calculate the bearing rate between adjacent GPS points by determining the absolute difference between their respective bearings.
    
    The SMF2016 dataset provides point-level bearing information. To calculate the bearing rate, we use the following equation (\ref{eq_br}):
    
    \begin{equation}
        BR_{P_m} = |\vn{bearing}(P_n) - \vn{bearing}(P_m)| \label{eq_br}
    \end{equation}

    It is noteworthy to mention also that the Geolife dataset does not include information on the bearing of location points. Therefore, in order to use \cref{eq_br}, we first derive point's bearing using \cref{geo_br1} through \cref{geo_br4} to calculate this missing information. The values of $\vn{lat}$ and $\vn{lon}$ are first transformed to radian before applying these equation. The result, converted to degrees, is then used in Equation \ref{eq_br} to calculate the bearing rate.

    \begin{align}
        y  &= \vn{sin}[P_n(\vn{lon})-P_m(\vn{lon})] \times \vn{cos} [P_n(\vn{lat})] \label{geo_br1} \\
    x &= \vn{cos}[P_m(\vn{lat})] \times \vn{sin}[P_n(\vn{lat})] - \vn{sin}[P_m(\vn{lat})] \times \nonumber %\label{geo_br} 
     \vn{cos}[P_n(\vn{lat})] \times \\ 
     & \vn{cos}[P_n(\vn{lon}) - P_m(\vn{lon})] \label{geo_br3}  \\
     \vn{bearing}(P_m) &= \vn{arctan(y,x)}  \label{geo_br4}
    \end{align}

    Ultimately, each location point in a segment is now represented by a vector consisting of four attributes, namely $<S_{P_i}, A_{P_i}, J_{P_i},  BR_{P_i}>$. 


    
    \subsection{Data split}

\section{Classification Algorithms Used}

\section{Establishin Baseline}


