\chapter{State of the Art} \label{chap2}

This chapter presents a broad overview of the topic: transportation mode detection (TMD). 

\section{Data Sources}

\subsection{Mobile Phone Network Data}

    The base transceiver stations (BTSs), commonly referred to as cell towers, constitute a fundamental component of cellular networks. Equipped with multiple antennas, these stations facilitate wireless communication between the network and mobile phones. A single BTS typically covers a limited geographical area, known as a cell.
    A group of BTSs within a specific location area (LA) is managed by a base station controller (BSC). The BSC oversees various network operations, including call setup and handover procedures, ensuring seamless communication as mobile devices move between cells \cite{huang2019transport}.

    Mobile Switching Centres (MSCs) play a crucial role in managing and coordinating multiple BSCs. MSCs establish and terminate end-to-end connections between mobile devices, as well as facilitating communication with external networks. MSCs handle the complexities of mobility management and handover procedures, ensuring uninterrupted service during device usage. An integral component of the MSC is the visitor location register (VLR), a database that maintains real-time records of the locations of all mobile subscribers within its coverage area. This is illustrated in \Cref{fig:mobile-phone-data}).
    
    \begin{figure}
        \centering        \includegraphics[width=0.8\textwidth]{figures/mobile-phone-data.png}
        \caption{An illustration of the mobile network infrastructure and network event handover. \cite{huang2019transport}.} 
        \label{fig:mobile-phone-data}
    \end{figure}
    
    Mobile network operators continuously track device locations to maintain service quality, estimating device positions based on the nearest BTS connection \cite{bachir2018combining}. While location precision varies with tower density, this approach enables large-scale, passive data collection without user intervention.
    Mobile phone network data can be broadly categorized into two types: event-driven and network-driven \cite{huang2019transport}. \emph{Event-driven data} is generated by user activity, such as phone calls and SMS messages, e.g. call detail records (CDRs) \cite{demissie2016inferring}. \emph{Network-driven data} is collected regardless of user activity and is triggered by events like location updates or handovers \cite{chen2016promises, phithakkitnukoon2017inferring}. Network operators utilize network-driven data to ensure optimal service quality for their subscribers.

    Passively collected mobile phone network data offers a distinct advantage. Users do not require active participation, potentially enabling the study of even hard-to-reach populations. However, mobile phone network data is not easily accessible. Due to privacy concerns and potential business competition, mobile network operators (MNOs) often hesitate to share data with researchers. Furthermore, mobile phone network data is characterised by poor spatio-temporal granularity. The temporal granularity of mobile data depends on the type used. Network-driven data is often denser, as it is collected irrespective of phone usage but is not recorded at regular intervals. Regarding spatial accuracy, mobile data is often less precise, as the location data is the location of the BTS, not the exact phone location.

    Several studies have explored the use of cellular data for TMD with varying degrees of success. For instance, Bachir et al. \cite{bachir2018combining} proposed an unsupervised method capable of classifying trips into road and rail categories. However, this method's performance is limited by its inability to handle multi-modal and uncertain journeys. Qu et al. \cite{qu2015transportation} proposed a transportation mode split model to estimate the shares of three transportation modes within each census tract. Nevertheless, the model's predictive accuracy can vary across different regions. Li et al. \cite{li2017public} proposed a novel public TMD system that accurately detects subway, train, and bus trips. However, the system's accuracy is contingent on the quality and coverage of cellular data, potentially limiting its performance in remote areas, during rapid user movement, or in crowded environments.

 

\subsection{Inertial Measurement Sensor Data}

    Inertial measurement unit (IMU) sensor data comprises datasets collected from wearable sensors attached to the human body. Typically, these sensors include accelerometers, gyroscopes, and magnetometers, providing highly detailed and complex data. Such datasets offer significant potential for the development and refinement of machine learning models \cite{drosouli2021transportation}. This data can be utilised to identify the user's current mode of transportation, such as walking or public transport \cite{wang2019enabling}.  
    However, IMU sensor data is not without limitations. Data acquisition is often costly and time-consuming, limiting participation and potentially compromising the representativeness of the dataset. Additionally, the placement of sensors on specific body locations can influence data quality and may not accurately reflect people's movement patterns in reality \cite{wang2019enabling}.

%    \begin{figure}
%        \centering        \includegraphics[width=0.75\textwidth]{figures/shl-data-collection.png}
%        \caption{A participant outfitted with several smartphones and camera for the SHL data acquisition \cite{wang2019enabling}.} 
%        \label{fig:imu-data1}
%    \end{figure}

    \begin{figure}[ht]
    \centering
    % Subfigure for SHL dataset
    \begin{subfigure}[b]{0.48\textwidth}
        \centering
        \includegraphics[width=\textwidth]{figures/shl-dataset.png}
        \caption{SHL dataset.}
        \label{fig:shl-data}
    \end{subfigure}
    \hfill
    % Subfigure for CAPTIMOVE dataset
    \begin{subfigure}[b]{0.48\textwidth}
        \centering
        \includegraphics[width=\textwidth]{figures/captimove-dataset.png}
        \caption{CAPTIMOVE dataset.}
        \label{fig:captimove-data}
    \end{subfigure}
    \caption{Participants outfitted with several smartphones and cameras for data acquisition in: (a) SHL dataset \cite{gjoreski2018university}, (b) CAPTIMOVE dataset \cite{alaoui2021urban}.}
    \label{fig:imu-data}
\end{figure}

    Recent studies have explored the use of IMU sensor data for TMD. Wang et al. \cite{wang2019enabling} proposed a standardised framework for transportation mode recognition using smartphone sensors. However, the study's limitations include a limited number of participants and geographic diversity in data collection. Subsequent research \cite{wang2020summary} indicated the superiority of deep learning methods over classical machine learning approaches, with the top-performing submission achieving an F1 score of 88.5\%. Nevertheless, this study also suffered from limited participant diversity, as data was collected from only two smartphone users.
    Hasan et al. \cite{hasan2022transportation} investigated the use of smartwatch and smartphone data to predict transportation mode. The study identified notable disparities in physical attributes and activity metrics associated with different modes of transportation. In a related study, Shin et al. \cite{shin2015urban} presented a method for urban data collection, sensing, and classifying diverse transportation activities using smartphone sensors. This approach enables the classification of transportation modes directly on smartphones.

\subsection{Global Positioning System Data}
    The Global Positioning System (GPS) is a satellite-based radio navigation system owned by the United States Space Force \cite{gpsgov}. It is a network of 24 or more satellites orbiting the Earth, and providing continuous global positioning, navigation, and timing (PNT) services \cite{hein2020status}.
    Each satellite emits radio signals containing information about its precise location, status, and time. GPS receivers on Earth detect these signals and calculate the distance between the receiver and each satellite using the known speed of light and the measured signal travel time. By triangulating the distances from multiple satellites, the receiver can determine its exact geographic coordinates, including latitude, longitude, and altitude.

    The increasing prevalence of GPS-enabled devices, such as smartphones, has led to the development of studies that focus on analysing mobility patterns and behaviour based on mobility data \cite{shen2014review}, including the identification of modes of transport. When satellites are visible, GPS data can be obtained with precise spatial and temporal accuracy.     
    The widespread availability of smartphones with sensing capabilities has enabled a cost-effective approach to obtaining GPS trajectory data for TMD studies. In addition to facilitating a cost-effective data collection, mobile crowdsensing enables this data to be collected at a high granularity, such as every 1 second in \cite{rodrigues2014sensemycity}, and at sampling rates of around 3-5 seconds in the GeoLife project \cite{zheng2010geolife}. The GeoLife project pioneered the TMD studies from GPS trajectory data in 2008 \cite{zheng2008geolife}. Since then, GPS data have been used to estimate mode of transport use by leveraging on the accuracy of GPS technology, as evidenced by the increasing number of publications \cite{dabiri2018inferring,  muhammad2021transportation, muhammad2023inferring,zheng2008learning, zheng2008understanding} on the topic in recent years. 

   % Advances in data mining and machine learning technologies further escalate research efforts utilising the vast data gathered through GPS-enabled devices for travel mode inference. Many researchers have employed GPS-enabled devices to conduct data collection for mode detection studies. For example, Bolbol et al. \cite{bolbol2012inferring} introduced an SVM-based method for mode detection, highlighting the significance of speed and acceleration as critical features. They evaluated the discriminative power of several features for six modes. Stenneth et al. \cite{stenneth2011transportation} developed a TMD framework that integrates GPS data with knowledge of the transportation network. Their framework achieves state-of-the-art accuracy on a small dataset of GPS trajectories using several machine-learning classification algorithms. Still, it is limited by the size and diversity of the dataset.

    Recent advancements in data mining and machine learning have intensified research efforts in utilising the vast quantities of data collected by GPS-enabled devices for travel mode inference. Numerous researchers have employed GPS-enabled devices to gather data for mode detection studies. For instance, Bolbol et al. \cite{bolbol2012inferring} proposed an SVM-based method for mode detection, emphasising the importance of speed and acceleration as key features. They evaluated the discriminative power of various features for six different modes of transportation. Stenneth et al. \cite{stenneth2011transportation} developed a TMD framework that integrates GPS data with transportation network knowledge. Their framework achieves state-of-the-art accuracy on a relatively small dataset of GPS trajectories using various machine learning classification algorithms. However, the framework's performance is limited by the size and diversity of the dataset.

    In recent years, the breakthrough performance of deep neural networks in various research domains, such as computer vision and natural language processing, has prompted its utilization in the context of TMD \cite{james2020travel}. Endo et al. \cite{endo2016deep} introduce a novel approach that models raw GPS trajectory data into 2D image structures. This transformation automatically extracts high-level features using a fully connected deep neural network. The core idea involves representing trajectory data as images, where pixel values represent the user's duration at specific locations. These image-based features are integrated with manual features and fed into a classical classifier for transportation mode prediction.  However, a limitation of this approach is that it solely relies on duration time and lacks consideration of spatio-temporal information. Dabiri and Heaslip \cite{dabiri2018inferring} propose using convolutional neural networks (CNNs) for mode detection from GPS trajectories. The study autonomously extracts high-level features from raw GPS data by leveraging CNN architectures and identifying five travel modes. This approach stands out for its ability to operate solely on GPS data, eliminating the need for contextual information. 
    
\section{Trajectory Data Mining}
     \subsection{Terminology}
     This section establishes the foundational definitions that will be consistently referenced throughout this thesis.

     \begin{definition}[\textbf{Point}]
        A point $P_i$ is a tuple \textit{(time, latitude, longitude)} representing a specific spatio-temporal location. It signifies the position of an object at a given timestamp.
    \end{definition}

    \begin{definition}[\textbf{Movement track}]
        The \textit{movement track} of a moving object is the temporally ordered sequence of spatio-temporal data points records as captured by a positioning device.
    \end{definition}
    Each data point record holds the exact instant of the capture and may include additional attributes captured by the device, such as instantaneous speed, altitude, and direction \cite{spaccapietra2013trajectories}. No two records share the exact instant value.

    \begin{definition}[\textbf{Trajectory}]
        A trajectory denotes a portion in the path traversed by a moving object within a defined time interval. It constitutes a sequence of spatio-temporal positions, each uniquely timestamped, thereby providing a discrete record of the object's successive locations throughout its movement.
    \end{definition}
    Formally, a trajectory $T$ can be represented as an ordered sequence $T = \{ \langle p_1, t_1 \rangle, \langle p_2, t_2 \rangle, \dots, \\ \langle p_n, t_n \rangle \} $, where each element  $\langle p_i, t_i \rangle $ indicates the object’s presence at location $p_i$ at time $t_i$ \cite{feng2016survey}. The timestamps are sequentially ordered (i.e., $t_j < t_k \quad \text{for} \quad 1 \leq j < k \leq n$), capturing the continuous movement of the object over the interval from the start time to the end time. 
%%    Throughout this thesis we will the terms \textit{trajectory} and \textit{trip} interchangeably to mean the same thing. 

    \begin{definition}[\textbf{trajectory segment}]
        A \textit{trajectory segment} refers to a specific portion or subsection of a trajectory.  It represents a continuous part of this path, usually defined by a specific time intervals [$t_{\text{begin}}, t_{\text{end}}$ ].
    \end{definition}
    In the context of transportation or movement analysis, a trajectory typically represents the path or course taken by an object or individual over time. \Cref{fig:trajectory} illustrates a portion of an object's movement trajectory, highlighting two specific segments that are deemed relevant for further  analysis.  
    \begin{figure}
        \centering
        \includegraphics[width=0.85\linewidth]{figures/trajectory.png}
        \caption{ Trajectory segments derived from the overall movement pattern of an object \cite{spaccapietra2013trajectories}.}
        \label{fig:trajectory}
    \end{figure}
    
%%%     \subsection{Trajectory Representation}
     
\section{Mode Detection Approaches} %Approaches to Mode Inference
    Over the past two decades \cite{reddy2008determining,stopher2005processing, zheng2008learning}, researchers have shown increasing interest in identifying transportation modes used by commuters, employing a diverse range of techniques, from statistical and expert system methods to more recent machine learning approaches. This section provides a brief overview of these approaches.

    \subsection{Ruled-based Approaches}
    Rule-based expert systems constitute a significant portion of TMD research. These systems rely on a predefined set of rules, often expressed in an \textquotedblleft \emph{if-then}\textquotedblright~ format. By analysing motion attributes such as average and maximum speed, proximity to network infrastructure (e.g., bus stops, train stations), or deviation from major roads, these systems effectively detect and classify commuters’ modes of transportation.
    Early research in this area, as exemplified by Bohte and Maat \cite{bohte2009deriving} and Shin et al.\cite{shin2015urban}, focused on the application of classical rule-based systems. More recent work, such as that of Xiao et al. \cite{xiao2019detecting}, has extended these techniques to address complex urban scenarios.
    
    To enhance the flexibility and adaptability of rule-based systems, researchers have incorporated fuzzy logic and membership functions. This approach, as demonstrated in \cite{biljecki2013transportation,das2018fuzzy,schuessler2009processing}, allows for more nuanced decision-making by considering the degree to which a trajectory exhibits anomalous characteristics. By combining the precision of rule-based systems with the flexibility of fuzzy logic, these methods provide a robust framework for TMD and analysis.

    \subsection{Statistical Methods}
    Statistical methods have been employed to infer transportation modes based on probabilistic models. Stopher et al. \cite{stopher2008search} proposed a probability matrix method that leverages speed characteristics to assign weights to different modes of transportation. However, this method struggles to distinguish between bus and car trips, especially under inconsistent speed data. A study by  Xu et al. \cite{xu2011transportation} introduced a probabilistic approach that combines a Hidden Markov Model (HMM) with Speed Distribution Law (SDL) and Cumulative Prospect Theory (CPT) to identify transportation modes using mobile phone data. This method aims to improve accuracy across various traffic conditions.

    More recently, the authors of \cite{wang2023bayesian} proposed a Bayesian Network Learning Framework to identify travel modes using cellular signaling data. This framework considers factors such as travel behavior, personal attributes, and traffic environment to accurately classify travel modes. Despite the potential of these statistical methods, their widespread adoption remains limited.
    
    \subsection{Machine Learning Methods}
    In recent years, machine learning models have emerged as the dominant approach in TMD research. These methods are broadly categorised into three primary branches: unsupervised and supervised learning. The following sections provide a brief overview of these approaches.
    
            \subsubsection{Unsupervised Learning Methods}
            Unsupervised learning \cite{flach2012machine} is a machine learning technique that trains models on unlabelled data. By analysing patterns within the data, unsupervised learning algorithms can perform tasks such as clustering, dimensionality reduction, and anomaly detection. In the context of TMD, unsupervised learning has been employed to identify transportation modes from mobility data.

            Bachir et al. \cite{bachir2018combining} utilised Bayesian inference combined with clustering techniques to classify transportation modes. Wang et al. \cite{wang2010transportation} employed k-means clustering to categorise trips based on their duration. This method divided trips into two clusters: one representing private vehicle trips and the other representing public transport trips. The cluster with a shorter average trip duration, aligning with Google Maps driving time estimates, was assigned to private vehicle trips, while the other cluster was assigned to public transport.

            Similarly, k-means clustering of cellular data is the primary technique used in \cite{kalatian2016travel} to discriminate between walking, public transport, and car modes. The authors aggregated clusters of neighbouring cells exhibiting data fluctuations and replaced these clusters with a weighted centre point.

            Lin et al. \cite{lin2013detecting} proposed a method based on the Kolmogorov-Smirnov test, utilising only instantaneous, average, and maximum speed metrics. The authors assumed that speed distributions across street segments remain consistent for vehicles of the same transport mode but differ significantly between modes, particularly in higher-speed ranges. While achieving high accuracy, the algorithm's performance was not benchmarked against other methods.

           More recently, Markos and Yu \cite{markos2020unsupervised} applied an unsupervised deep learning technique, DeepCAE, to identify transportation modes. By combining a pre-trained convolutional autoencoder with a clustering layer, their model learns meaningful representations from GPS trajectory data, enabling the discovery of underlying patterns and the classification of different transportation modes without requiring labelled training data.

           \subsubsection{Semi-supervised Learning Methods}
            Semi-supervised learning methods provide a solution to the challenge of limited labelled data availability in machine learning. Unlike unsupervised methods, which operate solely on unlabelled data, semi-supervised methods leverage both labelled and unlabelled data to enhance model performance. By combining these two data sources, semi-supervised methods can extract valuable information, thereby improving model accuracy and robustness.

            Recent years have witnessed a growing interest in semi-supervised approaches to TMD research. Dabiri et al. \cite{dabiri2019semi} proposed SECA, a novel semi-supervised deep learning approach for TMD utilising GPS trajectory data. This method combines a convolutional autoencoder with a convolutional neural network to extract relevant features from GPS segments. Their model outperformed state-of-the-art methods. However, a limitation of this study is the small number of users with labeled data, which was restricted to only two users. This limitation suggests that the algorithm's performance may deteriorate when labeled data is scarce.

            Building upon these advancements, James et al. \cite{james2020semi} proposed a semi-supervised deep ensemble learning approach for TMD that requires only a small amount of annotated data. This method leverages GPS trajectories of variable lengths and employs a tailored feature engineering process to construct a robust deep neural network architecture that accurately maps latent information to travel modes.
            
            Furthering the exploration of semi-supervised learning for TMD, Zhu et al. \cite{zhu2021semi} proposed a CNN-GRU model that leverages both labeled and unlabeled GPS trajectory data. By employing a pseudo-labeling technique, the model effectively extracts features from GPS trajectories. Additionally, a grouping-based aggregation scheme and a data flipping augmentation scheme are introduced to enhance the convergence and robustness of the framework. However, the model is susceptible to overfitting.
       

            
            \subsubsection{Supervised Learning Methods}
            The majority of machine learning methods applied to TMD are rooted in supervised learning. In this paradigm, models are trained on labelled data, where both input features (e.g., GPS trajectories or sensor data) and corresponding output labels (i.e., transportation mode) are provided. TMD is thus considered a supervised classification problem, with the labelled data serving as the ground truth for training and subsequent evaluation of the model.

            Pioneering studies exploring mobility patterns from data and subsequent transportation mode inference based on supervised learning were conducted by Zheng et al. \cite{zheng2008understanding, zheng2008learning} on the Geolife project. This project, which made the data openly available to enhance research reproducibility, focused on detecting four different transportation modes. Since then, a growing body of literature has emerged, with the vast majority utilising the Geolife dataset. Subsequent studies \cite{zheng2010understanding, zheng2010geolife} also employed a supervised learning approach.

            Building upon these initial studies, Stenneth et al. \cite{stenneth2011transportation} proposed a method to identify a user's mode of transportation using GPS data and real-time transportation network information, achieving high accuracy compared to previous methods. However, their approach relies on continuous input of transportation network data, limiting its applicability in certain scenarios. Bolbol et al. \cite{bolbol2012inferring} further demonstrated the applicability of supervised machine learning for TMD by segmenting trajectories, extracting features, and applying SVM for classification.

            A number of studies, \cite{de2023robust,jahangiri2015applying, xiao2017identifying} have employed supervised learning approaches for TMD. Jahangiri \cite{jahangiri2015applying} conducted a comparative study of various supervised learning methods, but the study was limited to specific sensors and transportation modes. Xiao et al. \cite{xiao2017identifying} proposed a hybrid approach using GPS data, segmenting trajectories, and extracting global and local features. To enhance model performance, they opted for tree-based ensemble models. Typically these studies rely on hand-crafted features engineered from raw data.

            Recent research has increasingly turned to deep learning architectures for TMD, leveraging their success in fields like computer vision and natural language processing. These models can automatically extract features and learn representations directly from raw data. For example, DeepTransport \cite{song2016deeptransport} employs a four-layer LSTM neural network to learn complex spatio-temporal patterns in human movement from large-scale heterogeneous data. The LSTM architecture is well-suited for this task due to its ability to capture temporal dependencies. By jointly learning from two sets of features, DeepTransport effectively models the spatio-temporal nature of human mobility.
            
            To further leverage the potentials of deep learning, some studies have explored the transformation of trajectory data into image-like representation, from which deep features are extracted. The proposal set forth in \cite{endo2016deep} attempts to overcome the limitations of hand-crafted features by employing deep feature extraction from these image-like representations. The final model is then trained in a supervised manner, utilising both these deep features and  hand-crafted features. Similarly, Ribeiro et al. \cite{ribeiro2024deep} utilise vision transformer-based neural networks to predict transportation modes directly from image representations of trajectory data, without the use of hand-crafted features.
            
            In contrast, other studies have focused on deep learning models that directly process raw sensor data without relying on hand-crafted features or image representations. Dabiri and Heaslip \cite{dabiri2018inferring} represented trajectory data in a 3D-like structure, suitable for a CNN-based models.  Drosouli et al. \cite{drosouli2023tmd} introduce TMD-BERT, a transformer-based model for TMD from sensor data. This model leverages the transformer architecture to process the entire sequence of sensor data, understanding the importance of each data point and assigning appropriate weights using attention mechanisms. This enables the model to capture global dependencies within the sequence, leading to more accurate predictions.

    \section{Transportation Mode Classification}
    This section provides a brief overview of the most common machine learning classification algorithms employed for TMD. As the primary focus of this thesis is on classification-based approach to TMD, our subsequent discussion will concentrate on ML classification algorithms.

    \subsection{k-Nearest Neighbours}
    The  k-Nearest Neighbours (k-NN) is a straightforward, distance-based  classification algorithm that operates on a lazy learning paradigm \cite{moreira2018general}. The k-NN lacks a dedicated training phase and instead memorises all training data, retaining it in memory. To predict the class of a new object, k-NN identifies the classes of the $k$ most similar objects. Relying solely on the class information from these closest neighbours, k-NN thus applies a local-learning approach. The parameter $k$ specifies the number of neighbouring labelled objects that the algorithm considers. A notable drawback of k-NN is its potentially slow classification of new objects, as it requires computing the distance between the new object and every object in the training set. This algorithm is used for mode detection in \cite{jahangiri2015applying,sadeghian2022stepwise}.

    \subsection{Decision Tree Algorithms}
    Decision trees (DTs) are a common supervised learning algorithm used for both classification and regression tasks \cite{moreira2018general}. They have been employed in TMD research, as evidenced by studies such as \cite{jahangiri2015applying, li2021transportation, stenneth2011transportation, zheng2008learning}. DTs construct a tree-like model where each internal node represents a \textquotedblleft test\textquotedblright~ on an attribute, each branch represents the outcome of that test, and each leaf node represents a class label. The algorithm recursively partitions the dataset based on the attribute that best separates the classes, using metrics such as information gain or Gini impurity to determine the optimal split. While DTs are intuitive to interpret and can model complex, non-linear relationships, they are prone to overfitting \cite{rokach2005decision}.

    \subsection{Support Vector Machines}
    Support Vector Machines (SVMs) are a family of supervised learning algorithms widely used for classification and regression tasks \cite{moreira2018general}. They are particularly effective in high-dimensional spaces and are known for their ability to handle complex nonlinear relationships between features and labels. The \textit{Support Vector Classifier (SVC)} is a popular variant of SVMs designed explicitly for classification problems. It aims to identify a hyperplane that effectively separates data points belonging to different classes while maximising the margin between the hyperplane and the nearest data points (called the \textit{support vectors}). This margin maximisation principle helps SVMs to achieve high classification accuracy and generalisation capabilities, even when dealing with noisy or imbalanced datasets \cite{wang2010boosting}. This algorithm is used in transportation mode classification in \cite{bolbol2012inferring,chen2024two, zheng2008understanding}. The computational cost of SVMs can be significant, especially for large datasets, as it depends on the number of support vectors. Additionally, SVM performance is highly sensitive to the choice of hyper-parameters.

    \subsection{Logistic Regression}
    Logistic Regression (LR) is a probability-based \cite{moreira2018general} supervised learning algorithm used for classification tasks. In LR, the goal is to model the probability that a given input (e.g., a set of features) belongs to a particular class. The model uses the logistic function \cite{kleinbaum2002logistic} (also known as the sigmoid function) to map predicted values to probabilities between 0 and 1:

    \begin{equation}
        P(Y=1 \mid X) = \frac{1}{1 + e^{-(\beta_0 + \beta_1 X_1 + \beta_2 X_2 + \cdots + \beta_n X_n)}}
    \end{equation}

where:
\begin{itemize}
    \item \( P(Y=1 \mid X) \) is the probability of the positive class given the input features \( X = (X_1, X_2, \dots, X_n) \),
    \item \( \beta_0 \) is the intercept term, and
    \item \( \beta_1, \beta_2, \dots, \beta_n \) are the coefficients for each feature.
\end{itemize}

    LR fits these coefficients by maximising the likelihood function, estimating parameters that best explain the observed data. It is commonly used in fields like medical research, social sciences, and machine learning for binary and multiclass classification tasks, due to its interpretability and computational efficiency. In the context of TMD, studies such as \cite{endo2016deep} have employed LR for classification.

    \subsection{Bayesian Belief Network}
    The Bayesian Belief Network (BN) is a probability-based graphical model that represents a domain variable (i.e. input features) and their probabilistic dependency using a directed acyclic graph (DAG). In this model, the variables are represented by nodes and the probabilistic relationships between them represented by edges \cite{cheng2001learning}. BN is widely used due to its ability to model complex dependencies and uncertainty in data.

    Each node within a BN is associated with a conditional probability distribution (CPD) that specifies the probability of its state given the states of its parent nodes. The BN learning process involves two main steps: structure learning and parameter learning. In structure learning, the network's structure is inferred by identifying conditional independence relationships between variables \cite{bouckaert1994properties}. Once the structure is learned, the parameters of the CPDs are estimated from the training data using maximum likelihood estimation or Bayesian methods.

    In a classification problem, given a set of input features $X$, a BN model can be used to predict the target variable $Y$. The model computes the posterior probability $P(Y|X)$ using Bayes' theorem:
    \begin{equation}
        P(Y \mid X) = \frac{P(X \mid Y) P(Y)}{P(X)}
    \end{equation}

    
where:

\begin{itemize}
    \item $P(Y)$ is the prior probability of the target class.
    \item $P(X|Y)$ is the likelihood of observing the data given the class.
    \item $P(X)$ is the marginal probability of the data, which can be ignored in classification as it remains constant across all classes.
\end{itemize}
    The class with the highest posterior probability is assigned to the instance. Several TMD studies have employed BN models for mode identification, including \cite{feng2013transportation,stenneth2011transportation, zheng2008learning}. A recent study by Wang et al. \cite{wang2023bayesian} also utilised BN for TMD.

    \subsection{Extreme Gradient Boosting}
   Extreme Gradient Boosting (XGBoost) is a powerful ensemble learning technique that builds on the principles of the gradient boosting algorithm \cite{chen2015higgs}. XGBoost constructs an ensemble of decision trees sequentially, adding trees that focus on correcting the errors made by previous models. This process is guided by a gradient descent-like optimisation algorithm, minimising a specific loss function. To enhance the model's generalisation performance, XGBoost incorporates regularisation techniques, tree pruning, and parallel computing. This distributed architecture allows for efficient training of large models on multiple machines. XGBoost has been successfully applied to TMD in studies such as \cite{alam2023federated,li2021transportation,xiao2017identifying}.

   \subsection{Random Forest}
% 1. Random forests (RF) is an ensemble method that combines multiple decision trees to enhance predictive accuracy. Each decision tree within the forest is trained on a different bootstrap sample \cite{moreira2018general,biau2016random}, similar to bagging; however, at each node, the split rule is selected from a random subset of predictive features rather than the full set. This randomisation approach is particularly practical for datasets with many predictive attributes, offering robust predictive performance and moderate interpretability. Although computationally intensive, random forests leverage decision trees as base learners and are versatile for classification and regression tasks. The prediction variability decreases as more trees are added to the ensemble, improving result stability.

    Random Forest (RF) is a powerful ensemble learning technique that constructs multiple decision trees and aggregates their predictions. Each tree is trained on a different bootstrap sample of the training data \cite{biau2016random,moreira2018general}, introducing randomness and reducing overfitting. Additionally, at each node of a tree, a random subset of features is considered for the split, further enhancing diversity. This feature selection process helps to prevent individual trees from becoming overly reliant on a single feature, improving the overall model's robustness.

    By combining multiple diverse trees, RF can often achieve higher accuracy and better generalisation performance compared to individual decision trees. The final prediction is determined through a majority vote for classification tasks. This ensemble approach leads to more robust and stable predictions, making RF a popular choice for various machine learning applications, including TMD \cite{alam2023federated, giri2022application, muhammad2023inferring, wang2017travel}.


    \subsection{Deep Learning Algorithms}
    Deep learning algorithms have achieved significant breakthroughs in various fields, including computer vision, image processing, and natural language processing \cite{dong2021survey, wang2020deep}. As a subfield of machine learning, deep learning utilizes artificial neural networks to learn complex patterns from large datasets. This capability has enabled advancements in tasks such as image recognition, natural language processing, and speech recognition.

    Recent advancements in data availability, particularly the proliferation of sensor data, and computational power, such as the development of GPUs, have contributed to the widespread adoption of deep learning techniques \cite{khan2019future}. Inspired by these developments, researchers have begun to apply deep learning techniques to the domain of spatio-temporal data mining, specifically in the context of TMD. The most common deep learning architectures used in this area are Convolutional Neural Networks (CNNs) and Long Short-Term Memory (LSTM) networks.

    
%    >>>Deep learning architectures have achieved significant breakthroughs in various fields, including computer vision, image processing, and natural language processing \cite{dong2021survey, wang2020deep}. Deep learning algorithms, a sub-branch of machine learning, utilise artificial neural networks to learn complex patterns from large datasets. This capability has enabled advancements in tasks such as image recognition, natural language processing, and speech recognition. The drivers to wide use of deep learning in recent years \cite{khan2019future} are mainly due to data availability (e.g big sensor data) and computational scale such as the development of the graphics processing units(GPU). Inspired by these successes, researchers have begun to apply deep learning techniques to the domain of spatio-temporal data mining, specifically in the context of TMD. The most common deep learning architectures used in this area are Convolutional Neural Networks (CNNs) and Long Short-Term Memory (LSTM) networks.

    \subsubsection{Convolutional Neural Networks}
    Convolutional Neural Networks (CNN) are a class of deep learning models specifically designed for processing structured grid data, such as images \cite{Aggarwal2018ann}. CNN utilise convolutional layers to extract spatial hierarchies of features, starting from simple patterns like edges and progressing to more complex structures. A typical CNN architecture comprises several layers: convolutional, pooling, and fully connected. Convolutional layers apply filters to the input data (e.g., an image) to detect local patterns like edges, textures, and shapes. Pooling layers reduce the spatial dimensions of the input features, preserving essential features and reducing computational complexity. Finally, fully connected layers connect the extracted features to the output layer, making predictions about the input data.

    While CNN are primarily associated with image data, they are also well-suited for processing spatio-temporal data \cite{Aggarwal2018ann}. Several TMD classification models, such as those in \cite{dabiri2018inferring, fan2024multi, liang2017convolutional, ma2023multi}, are primarily built using CNN architectures.

    \subsubsection{Long Short-Term Memory}
    Long Short-Term Memory (LSTM) networks are a specialised type of recurrent neural network (RNN) designed to process sequential data and address challenges associated with long-term dependencies. By incorporating advanced gating mechanisms and memory cells, LSTM regulate the flow of information within the network \cite{schmidhuber1997long}. These gates enable the selective retention or forgetting of past data, allowing the network to focus on relevant patterns over extended sequences. This design equips LSTM to effectively capture temporal patterns in data, such as time-series data, while mitigating the issue of vanishing gradients.

    LSTM networks have been widely adopted in TMD research due to their ability to capture temporal dependencies in sequential data. Studies such as \cite{asci2019novel, ding2024hybrid, drosouli2021transportation,  liu2017end, song2016deeptransport} have successfully employed LSTM networks for TMD tasks, leveraging their capability to model complex temporal patterns in GPS trajectories and other sensor data.

    
    \section{Features for Mode Detection}
    Feature engineering is a critical step in TMD, as it involves extracting relevant information from raw data to enhance model performance. A diverse range of features have been used. The features that have proven effective for TMD tasks can be primarily categorised as kinematic, spatial, and contextual \cite{sowlati2024approach}.
    
    Kinematic features, derived from the spatial and temporal information embedded in trajectory data, relate to the motion of the subject. The majority of previous studies have primarily relied on kinematic features. Features like speed \cite{bantis2017you,biljecki2013transportation,kalatian2016travel}, acceleration \cite{liang2017convolutional,liang2019deep,xiao2019detecting}, jerk \cite{dabiri2018inferring,dabiri2019semi,li2020coupled}, gyroscope's angular velocity \cite{alam2023feature, gjoreski2020classical}, magnetometer \cite{taherinavid2023automatic, xu2024automatic} and rotation vector data \cite{ashqar2018smartphone}, along with their statistical derivatives (e.g. mean and standard deviation of speed), are the dominant features in the literature.
    
    Spatial features focus on the spatial characteristics of the data, such as geographic coordinates. This category includes features like total distance travelled \cite{dabiri2019semi, wang2017travel}, heading change, movement direction \cite{pei2024travel, xiao2017identifying} and stop rate \cite{erdelic2019classification, zheng2008understanding}. These features provide insights into the spatial patterns of movement and can help distinguish between different transportation modes. For instance, the total distance travelled can help differentiate between walking and driving, while heading changes and movement direction can provide clues about turning manoeuvrers and changes in speed.

    Contextual features provide additional information about the environment and conditions surrounding the trip. These can include features such as the day of the week \cite{nawaz2020mode}, time of day \cite{zhu2016identifying}, trip duration \cite{andrade2022you}, and external data sources such as road and rail network data \cite{bachir2018combining}, proximity to stations or bus stops \cite{zeng2023trajectory, zeng2024travel}, GIS layers \cite{hong2023evaluating, roy2022assessing, sadeghian2022stepwise}, and transportation network data \cite{song2016deeptransport, wang2023bayesian}.
    
    The careful selection and engineering of features significantly impacts the performance of TMD models. By considering the specific characteristics of the dataset and the desired level of model accuracy, researchers can optimise feature engineering strategies to achieve superior results. \Cref{tab:tmd-survey} provides a summary of previous studies on this subject.

   
%\newpage     
%%%% landscape for long table %%%
%\pagestyle{plain}
%\begin{landscape}
%\begin{table}[h]
%   \centering
%   \caption{Overview of approaches used in transport mode detection research} \label{tab:tmd-survey-draft}
%   \begin{tabular}{lc*{11}{c}} 
%     \toprule
%     Study &Approach &Data source &Participants &Duration &Granularity &No. of trips &External data &Features &Classifier &No. of modes &Accuracy \\
%     \midrule
%    \cite{zhang2021toward} & & & & & &   &  &  speed, acceleration, \& jerk.& &   & \\
%      & & & & & &   &  & & &   & \\
%      & & & & & &   &  & & &   & \\    
%     \bottomrule
%   \end{tabular}
%\end{table}
%\end{landscape}
%\pagestyle{fancy}


%\newpage
%%%% Landscape for long table %%%
%\pagestyle{plain}
%\begin{landscape}
%\begin{adjustwidth}{-1cm}{-1cm} % Reduce left and right margins for the landscape page
%\small % Reduce font size
%\begin{longtable}{lp{2.5cm}p{2cm}p{1.5cm}p{1.5cm}p{1.5cm}p{1.5cm}p{1.5cm}p{3cm}p{2cm}p{1.5cm}p{1.5cm}}
%\caption{Overview of approaches used in transport mode detection research} \label{tab:tmd-survey} \\ 
%\toprule
%Study & Approach & Data source & Participants & Duration & Granularity & No. of trips & External data & Features & Classifier & No. of modes & Accuracy \\
%\midrule
%\endfirsthead
%\toprule
%Study & Approach & Data source & Participants & Duration & Granularity & No. of trips & External data & Features & Classifier & No. of modes & Accuracy \\
%\midrule
%\endhead
%\bottomrule
%\endfoot
% #1
%\cite{zheng2008learning} & supervised & GPS data & 45 & 6 months & 1-5 s &  &  No & distance, velocity, acceleration \& statistics & DT & 4 & 72.8\% \\
% #2
%\cite{schuessler2009processing} & rule-based & GPS data  & 4,882 & -- & 5-10 m &  & No & speed and acceleration statistics & fuzzy engine & 5 & -- \\

% &  &  &  &  &  &  &  &   &  &  &  \\
 
% &  &  &  &  &  &  &  &  &  &  &  \\

%\cite{zhang2021toward} & ... & ... & ... & ... & ... & ... & ... & speed, acceleration, \& jerk. & ... & ... & ... \\

% &  &  &  &  &  &  &  &  &  &  &  \\
%\end{longtable}
%\end{adjustwidth}
%\end{landscape}
%\pagestyle{fancy}


\newpage 
\pagestyle{plain}
\begin{landscape}
\begin{adjustwidth}{-1.5cm}{-1.5cm} % Further reduce left and right margins
\small % Reduce font size
\begin{longtable}{>{\centering\arraybackslash}p{1cm} >{\centering\arraybackslash}p{2.5cm} >{\centering\arraybackslash}p{1.8cm}
>{\centering\arraybackslash}p{1.7cm} >{\centering\arraybackslash}p{1.5cm} >{\centering\arraybackslash}p{1.3cm}
>{\centering\arraybackslash}p{1.cm} >{\raggedright\arraybackslash}p{6.5cm} 
>{\centering\arraybackslash}p{2.2cm} >{\centering\arraybackslash}p{1.cm} >{\centering\arraybackslash}p{1.cm}}
\caption{Overview of approaches used in transport mode detection research} \label{tab:tmd-survey} \\ 
\toprule
Study & Approach & Data source & Participants & Duration & Granularity & External data & Features & Classifier & No. of modes & Accuracy (\%)\\
\midrule
\endfirsthead
\toprule
Study & Approach & Data source & Participants & Duration & Granularity & External data & Features & Classifier & No. of modes & Accuracy (\%) \\
\midrule
\endhead
\bottomrule
\endfoot
% #1
\cite{zheng2008learning} & Supervised & GPS data & 45 & 6 months & 1-5 s & No & distance, velocity, acceleration \& statistics & DT & 4 & 72.8 \\
% #2
\cite{schuessler2009processing} & Rule-based & GPS data & 4,882 & -- & 5-10 m & No & speed and acceleration statistics & fuzzy engine & 5 & -- \\
% #3
\cite{stenneth2011transportation} & Supervised & GPS data & 6 & 3 weeks & 15 s & GIS & speed, acceleration, travel direction, closeness to station  & RF & 6 & 93.7 \\
% #4
\cite{bolbol2012inferring} & Supervised & GPS data & 81 & 2 weeks & 60 s & No & velocity and acceleration statistics  & SVM & 6 & 88 \\
% #5
\cite{biljecki2013transportation} & Rule-based & GPS data & 17M points & 7 days & 5-6.5 s & GIS & speed statistics  & fuzzy engine & 10 & 91.6 \\
% #6
\cite{zhang2021toward} & Semi-supervised & GPS data & 69 & > 5 years & 1-5 s & No & speed, acceleration, \& jerk. & DNN & 5 & 91.4 \\
% #7
\cite{huss2014using} & Statistical & GPS data & 12 & -- & 3.5-5 m & GIS & speed acceleration, rate-of-change & NPDA\dag  & 5 & 0.95\ddag   \\
% #8
\cite{shin2015urban} & Rule-based & IMU sensor & 30 & -- & 1 s & No & acceleration  & -- & 4 & 82 \\
% #9
\cite{jahangiri2015applying}  & Supervised & IMU sensor & 10 & -- & -- & No & acceleration, gyroscope, rotation vector & RF & 5 & 93 \\
% # 10
\cite{endo2016deep} & Supervised & GPS data &69 & > 5 years & 1-5 s & No & distance, velociy, deep features & LR & 7 & 67.9 \\
% 11
\cite{kalatian2016travel} & Unsupervised & Mobile data & 300,000 & 3 days & 5 min & No & speed statistics & K-means & 3 & -- \\
%
\cite{liang2017convolutional}& Supervised & IMU sensor & -- &--  &--  & No & acceleration & CNN & 7 & 94.5 \\
% 13
\cite{bantis2017you}& Unsupervised & GPS data & 2 & 3 days & 2 min & No & speed statistics & HMM & 4 & 78 \\
% 14
\cite{danafar2017bayesian}& Ruled-based & Mobile data & 3 & -- & -- & TN$\star$ & speed statistics & heuristic &6  & -- \\
% 15
\cite{xiao2017identifying} & Supervised & GPS data & 69 & > 5 years & 1-5 s & No & velocity, acceleration turning angle & XGB & 6 & 90.8 \\
% 16
\cite{dabiri2018inferring} & Supervised & GPS data & 69 & > 5 years & 1-5 s & No & speed, acceleration, jerk, bearing rate & CNN & 5 & 84.8 \\
% 17
\cite{dabiri2019semi} &Semi-supervised  & GPS data & 69 & > 5 years & 1-5 s & No & speed, acceleration, jerk, distance  &  convolutional autoencoder & 5 & 76.8  \\
% 18
\cite{bachir2018combining} & Unsupervised & Mobile data & -- & 1 month & -- & GIS & distance, number of roads/rail lines, number of station & bayesian inference & 2 & 0.96* \\
% 19
\cite{xu2011transportation} & Statistical & Mobile data & 500 & 20 days & -- & No & speed statistics & HMM & 3 & 91.7 \\
%%%%%%% key %%%%
\midrule
\multicolumn{11}{c}{\footnotesize{-- -not available \quad\quad  NPDA\dag - non-parametric discriminant analysis \quad\quad \ddag - Cohen’s kappa \quad TN$\star$} - transportation network \quad * - pearson correlation} \\
%%%%%% table break  %%%%%
% 20
\cite{wang2023bayesian} & Statistical & Mobile data & -- & -- & -- & GIS & distance, speed, personal attributes (age, gender) & bayesian network & 5 & 82.9 \\
%21
\cite{markos2020unsupervised} & Unsupervised & GPS data  & 69 & > 5 years & 1-5 s & No & speed, acceleration, jerk, bearing rate & convolutional autoEncoder & 5 & 80.5 \\
% #22
\cite{stopher2008search}& Statistical & GPS data & 72 & 56 days & 15-40 s  & GIS & speed statistics & probability matrix & 5 & 95 \\

%$\clubsuit$ & $\heartsuit$ & $\spadesuit$ & $\prod$ & $\langle$ & $\lfloor$ & $\uparrow$ &  &  &  & $\updownarrow$ \\

\end{longtable}
\end{adjustwidth}
\end{landscape}
\pagestyle{fancy}



\section{Concluding Remark}
This chapter has presented a comprehensive review of state of the art research on transportation mode detection. We have discussed various data types employed, different approaches to mode detection, and the application of supervised machine learning techniques. Additionally, we have explored the types of features used to enhance the accuracy of TMD models.


